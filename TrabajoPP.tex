\documentclass[12pt]{book}

\usepackage{amssymb}
\usepackage{amsmath}
\usepackage{amsthm}
\usepackage{amsfonts}
\usepackage[spanish]{babel}
\usepackage[utf8]{inputenc}
\usepackage[T1]{fontenc}
\usepackage{graphicx}
\usepackage[figuresright]{rotating}
\usepackage{subfigure}
\usepackage{epstopdf}
\usepackage{float}
\usepackage{natbib}
\usepackage[skip=10pt,labelfont=bf,labelsep=period]{caption}
\usepackage[paperwidth=215mm,paperheight=280mm,left=40mm,top=40mm,textwidth=150mm,textheight=215mm]{geometry} 
\usepackage{fancybox}
\usepackage{fancyhdr}
\usepackage{enumerate}
\usepackage{url} 
\usepackage{newtxtext,newtxmath}
\usepackage{bm}

\theoremstyle{definition}
\newtheorem{theorem}{Teorema}[chapter]
\newtheorem{example}[theorem]{Ejemplo}
\newtheorem{proposition}[theorem]{Proposición}
\newtheorem{definition}[theorem]{Definición}
\newtheorem{Lem}[theorem]{Lema}
\newtheorem{Cor}[theorem]{Corolario}
\newtheorem{remark}[theorem]{Observación}
\newtheorem{note}[theorem]{Nota}
\DeclareMathOperator{\sign}{sign}
\DeclareMathOperator*{\argmax}{\arg\,\max}
\usepackage{mathtools}
\newcommand{\Op}[3]{\prescript{}{#2}{#1}^{#3}_{t}}
\newcommand{\OpFull}[5]{\prescript{#1}{#2}{#3}^{#4}_{#5}}
\spanishdecimal{.}

%% Comandos basicos en el texto
%% ==================================================================
\newcommand{\set}[2]{\{#1\nonscript\;\vert\allowbreak\nonscript\:\mathopen{}#2\}}
\usepackage{dsfont}

\begin{document}
\begin{titlepage}

	\vfill

	{\LARGE Trabajo en teoría de Gráficas}\\[2cm]

	\vfill
\end{titlepage}
\section*{Definiciones}
En este trabajo se consideran gráficas simples, finitas, sin lazos.
\begin{definition}
Una gráfica $G$ consta de dos conjuntos $G=(V,E)$, donde $V$ es un conjunto cualquiera y $E\subset \set{\left\{u,v\right\}\subset V}{u\neq v}$. El conjunto $V$ o $V(G)$ es llamado conjunto de vértices de la gráfica $G$. Los elementos de $E$ o $E(G)$ se llaman aristas de la gráfica $G$.
\end{definition}

\begin{definition}
Sea $G=(V,E)$ una gráfica. Si $u,v\in V(G)$ son tal que $\left\{u,v\right\}\in E(G)$, decimos que $u$ y $v$ son adyacentes. Se denota tal adyacencia por $u\sim v$.
\end{definition}

\begin{definition}
Sea $G$ una gráfica. Una subgráfica de $G$ es una gráfica $H$ tal que $V(H)\subset V(G)$ y $E(H)\subset E(G)$.
\end{definition}

\begin{definition}
Sea $G$ una gráfica y $H$ una subgráfica de $G$. $H$ es una subgráfica inducida si para todo $u,v\in V(H)$ tales que $u\sim v$ en $G$ entonces $u\sim v$ en $H$.
\end{definition}

Se entiende que una subgráfica completa $H_n$ tiene cada par de sus $n$ vértices adyacentes. Dicha gráfica completa es maximal si no existe otro vértice en la gráfica tal que forme una completa más grande. Lo anterior da paso a la siguiente definición.

\begin{definition}[\citealt{Harary:1969}]
Un clan de $G$ es una subgráfica completa maximal. 
\end{definition}

\begin{definition}[\citealt{Roberts:1971}]
Dado $G$ una gráfica, sean $C_1, C_2, \dots, C_n $ sus clanes. Definimos $H'$ mediante $ V(H') = \{C_1, C_2, \dots, C_n\}$ y $\left\{C_i, C_j\right\}\in E(H')$ si y solo si $i \neq j$ y $C_i \cap C_j \neq \emptyset$.  
Entonces, llamamos a $H'$ como la gráfica de clanes de $G$ y escribimos $H'=K(G)$.
\end{definition}

\begin{definition}[\citealt{Alcon:2006}]
Sea $G$ una gráfica y $v \in V(G)$. Como es habitual, $G-v$ denota la gráfica inducida por $V(G)\setminus \{v\}$.  

El vértice $v$ es clan crítico si $K(G)\neq K(G-v)$. Una gráfica $G$ es clan crítica si cada uno de sus vértices es crítico.
\end{definition}



\section*{Resultados}

En el siguiente teorema se consideran las gráficas $G(4,3,3)$, que denota a la gráfica con 4 vértices, 3 aristas, número 3; y la gráfica $G(5,7,1)$, que denota la gráfica de 5 vértices, 7 aristas, número 1. Dichas gráficas son extraidas del apéndice de gráficas en \cite{Harary:1969}.
\begin{theorem}
	Sea $G$ una gráfica clan crítica tal que su gráfica de clanes es $K_3$, entonces $G$ es la gráfica $G(4,3,3)$ ó $G(5,7,1)$.
\end{theorem}
\begin{proof}
Sea $G$ una gráfica clan crítica tal que $K(G)=K_3$. Al ser $K(G)=K_3$ entonces existen $C_1,C_2,C_3$ clanes de $G$ tal que se intersecan de alguna manera.
Sea $u_1$ el vértice de $G$ en el que se intersecan necesariamente los tres clanes de $G$, esto es así ya que, de no serlo, no se obtiene $K_3$ como su gráfica de clanes de $G$ y al mismo tiempo ésta sea clan crítica.

Para dichas intersecciones, resultan los siguientes casos a considerar.
\begin{itemize}
\item Caso 1. 
Los clanes de $G$ se intersecan únicamente en el vértice $u_1$, es decir $C_1\cap C_2\cap C_3=\left\{u_1\right\}$ y $(C_1\cap C_3)\setminus C_2=\left\{\emptyset\right\}  \text{ y }(C_2\cap C_3)\setminus C_1=\left\{\emptyset\right\}$, como se muestra en la figura~\ref{F1}.

\begin{figure}[!htbp]
	\centering
	\includegraphics[scale=1.2]{Fig1.pdf}
	\caption{Esquema de los clanes de la gráfica $G$ correspondiente al caso 1.\label{F1}}
\end{figure}

De primera instancia, los tres clanes pueden ser tales que $|C_i|=n$ con $i=1,2,3$, considerando el vértice $u_1$ claramente; sin embargo, de considerar un $n>2$, la gráfica $G$ no cumpliría la condición de ser clan crítica, pues resultaría que $K(G)=K(G-\hat{u})$, para cualquier $\hat{u}\in V(C_i)$ y $\hat{u}\neq u_1$. Por lo tanto cada clan consta únicamente de dos vértices, uno de ellos $u_1$, con lo cual la gráfica $G=G(4,3,3)$, la cual cumple que $K(G)=K_3$.

\item Caso 2.
Existe una intersección dos a dos entre los clanes, de más de un vértice, considerando claramente el vértice $u_1$. Sin pérdida de generalidad, supongamos que el clan $C_3$ intersecta al clan $C_1$ en un vértice $u_2\in V(G)$, así como a $C_2$ en un vértice $u_3\in V(G)$, esto es 
\begin{equation*}
\left\{u_2\right\}=(C_1\cap C_3)\setminus C_2 \quad \text{y} \quad \left\{u_3\right\}=(C_2\cap C_3)\setminus C_1.
\end{equation*}
Como se muestra en la figura~\ref{F2}.

\begin{figure}[!htbp]
	\centering
	\includegraphics[scale=1.2]{Fig2.pdf}
	\caption{Esquema de los clanes de la gráfica $G$ correspondiente al caso 2.\label{F2}}
\end{figure}

Los clanes $C_1,C_2$ no pueden ser clanes constituidos únicamente de dos vértices, pues de serlo, $E(G)=\left\{\left\{u_2,u_1\right\},\left\{u_1,u_3\right\},\left\{u_2,u_3\right\}\right\}$, es decir $G=K_3$, lo que implica que $K(G)\neq K_3$. Se considera que $|C_3|=3$, pues si no fuese así, no se satisfce la condicion de que $G$ es clan crítica, pensando en que $K(G-\hat{u})=K(G)$ para todo $\hat{u}\in V(C_3)$ distintos a $u_1,u_2,u_3$. Usando un razonamiento similar, es claro que los clanes $C_1$ y $C_2$ también satisfacen la propiedad que $|C_1|=|C_2|=3$. Por lo tanto, $G=G(5,7,1)$ y es claro que $K(G)=K_3$.


\item Caso 3.
Todos los clanes se intersecan en el vértice $u_1$ y sólo existe otra intersección entre dos de estos clanes. Sin pérdida de generalidad supongamos que los clanes $C_1$ y $C_3$ se intersecan en un vértice $u_2$ y $(C_2\cap C_3)\setminus C_1=\left\{\emptyset\right\}$, como se muestra en la figura~\ref{F3}.

\begin{figure}[!htbp]
	\centering
	\includegraphics[scale=1.2]{Fig3.pdf}
	\caption{Esquema de los clanes de la gráfica $G$ correspondiente al caso 3.\label{F3}}
\end{figure}

Utilizando el mismo argumento del caso anterior, es fácil ver que $C_1= C_3=K_3$, pues de no ser así, no se satisface la hipótesis de ser $G$ clan crítica. Sea $u_3\in V(C_3)$ tal que $u_3\neq u_1,u_2$ y que satisface la inclusión $\left\{\left\{u_2,u_1\right\},\left\{u_1,u_3\right\},\left\{u_2,u_3\right\}\right\}\subset E(G)$. El clan $C_2$ debe estar constituido de más de dos vértices, pues de no ser así, consideremos un nuevo vértice $\hat{u}\neq u_1$ en $C_2$, ver figura~\ref{F4}.

\begin{figure}[!htbp]
	\centering
	\includegraphics[scale=1.2]{Fig4.pdf}
	\caption{Esquema de los vértices y clanes de la gráfica $G$ correspondiente al caso 3, suponiendo que $C_2$ cuenta únicamente con dos vértices.\label{F4}}
\end{figure}

En tal caso se cumple que $K(G-u_2)=K(G)$, situación que no es válida. Con lo cual $|C_2|>2$ y se consideran los siguientes dos casos.
\begin{itemize}
\item Caso 3.1.
El vértice $u_3\notin V(C_2)$. Se cumple la propiedad que para todo $\overline{u}\in V(C_2)$ distintos al vértice $u_1$, resulta que $K(G-\overline{u})=K(G)$, lo que implica que este caso no es posible.

\item Caso 3.2.
El vértice $u_3\in V(C_2)$. Este caso se reduce al caso 2, con lo cual $G=G(5,7,1)$ y por lo tanto $K(G)=K_3$.
\end{itemize}
\end{itemize}
Resulta entonces que $G$ es la gráfica $G(4,3,3)$ ó $G(5,7,1)$.
\end{proof}





\bibliography{Ref}
\bibliographystyle{apalike}

\end{document}