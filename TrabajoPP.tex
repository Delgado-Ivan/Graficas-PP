\documentclass[12pt]{book}

\usepackage{amssymb}
\usepackage{amsmath}
\usepackage{amsthm}
\usepackage{amsfonts}
\usepackage[spanish]{babel}
\usepackage[utf8]{inputenc}
\usepackage[T1]{fontenc}
\usepackage{graphicx}
\usepackage[figuresright]{rotating}
\usepackage{subfigure}
\usepackage{epstopdf}
\usepackage{float}
\usepackage{natbib}
\usepackage[skip=10pt,labelfont=bf,labelsep=period]{caption}
\usepackage[paperwidth=215mm,paperheight=280mm,left=40mm,top=40mm,textwidth=150mm,textheight=215mm]{geometry} 
\usepackage{fancybox}
\usepackage{fancyhdr}
\usepackage{enumerate}
\usepackage{url} 
\usepackage{newtxtext,newtxmath}
\usepackage{bm}

\theoremstyle{definition}
\newtheorem{theorem}{Teorema}[chapter]
\newtheorem{example}[theorem]{Ejemplo}
\newtheorem{proposition}[theorem]{Proposición}
\newtheorem{definition}[theorem]{Definición}
\newtheorem{Lem}[theorem]{Lema}
\newtheorem{Cor}[theorem]{Corolario}
\newtheorem{remark}[theorem]{Observación}
\newtheorem{note}[theorem]{Nota}
\DeclareMathOperator{\sign}{sign}
\DeclareMathOperator*{\argmax}{\arg\,\max}
\usepackage{mathtools}
\newcommand{\Op}[3]{\prescript{}{#2}{#1}^{#3}_{t}}
\newcommand{\OpFull}[5]{\prescript{#1}{#2}{#3}^{#4}_{#5}}
\spanishdecimal{.}

%% Comandos basicos en el texto
%% ==================================================================
\newcommand{\set}[2]{\{#1\nonscript\;\vert\allowbreak\nonscript\:\mathopen{}#2\}}
\usepackage{dsfont}

\begin{document}
\begin{titlepage}

	\vfill

	{\LARGE Trabajo en teoría de Gráficas}\\[2cm]

	\vfill
\end{titlepage}
\section*{Definiciones}
En este trabajo se consideran gráficas simples, finitas, sin lazos.
\begin{definition}
Una gráfica $G$ consta de dos conjuntos $G=(V,E)$, donde $V$ es un conjunto cualquiera y $E\subset \set{\left\{x,y\right\}\subset V}{x\neq y}$. El conjunto $V$ es llamado conjunto de vértices. Los elementos de E se llaman aristas.
\end{definition}




\section*{Resultados}

\begin{theorem}
	Sea $G$ una gráfica clan crítica tal que su gráfica de clanes es $k_3$, entonces $G$ es la gráfica 4,3 o 5,7.
\end{theorem}
\begin{proof}
Sea $G$ una gráfica clan crítica tal que $K(G)=k_3$.
\end{proof}











\bibliography{Ref}
\bibliographystyle{apalike}

\end{document}