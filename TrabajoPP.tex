\documentclass[12pt]{book}

\usepackage{amssymb}
\usepackage{amsmath}
\usepackage{amsthm}
\usepackage{amsfonts}
\usepackage[spanish]{babel}
\usepackage[utf8]{inputenc}
\usepackage[T1]{fontenc}
\usepackage{graphicx}
\usepackage[figuresright]{rotating}
\usepackage{subfigure}
\usepackage{epstopdf}
\usepackage{float}
\usepackage{natbib}
\usepackage[skip=10pt,labelfont=bf,labelsep=period]{caption}
\usepackage[paperwidth=215mm,paperheight=280mm,left=40mm,top=40mm,textwidth=150mm,textheight=215mm]{geometry} 
\usepackage{fancybox}
\usepackage{fancyhdr}
\usepackage{enumerate}
\usepackage{url} 
\usepackage{newtxtext,newtxmath}
\usepackage{bm}

\theoremstyle{definition}
\newtheorem{theorem}{Teorema}[chapter]
\newtheorem{example}[theorem]{Ejemplo}
\newtheorem{proposition}[theorem]{Proposición}
\newtheorem{definition}[theorem]{Definición}
\newtheorem{Lem}[theorem]{Lema}
\newtheorem{Cor}[theorem]{Corolario}
\newtheorem{remark}[theorem]{Observación}
\newtheorem{note}[theorem]{Nota}
\DeclareMathOperator{\sign}{sign}
\DeclareMathOperator*{\argmax}{\arg\,\max}
\usepackage{mathtools}
\newcommand{\Op}[3]{\prescript{}{#2}{#1}^{#3}_{t}}
\newcommand{\OpFull}[5]{\prescript{#1}{#2}{#3}^{#4}_{#5}}
\spanishdecimal{.}

%% Comandos basicos en el texto
%% ==================================================================
\newcommand{\set}[2]{\{#1\nonscript\;\vert\allowbreak\nonscript\:\mathopen{}#2\}}
\usepackage{dsfont}

\begin{document}
\begin{titlepage}

	\vfill

	{\LARGE Trabajo en teoría de Gráficas}\\[2cm]

	\vfill
\end{titlepage}
\section*{Definiciones}
En este trabajo se consideran gráficas simples, finitas, sin lazos.
\begin{definition}
Una gráfica $G$ consta de dos conjuntos $G=(V,E)$, donde $V$ es un conjunto cualquiera y $E\subset \set{\left\{u,v\right\}\subset V}{u\neq v}$. El conjunto $V$ o $V(G)$ es llamado conjunto de vértices de la gráfica $G$. Los elementos de $E$ o $E(G)$ se llaman aristas de la gráfica $G$.
\end{definition}

\begin{definition}
Sea $G=(V,E)$ una gráfica. Si $u,v\in V(G)$ son tal que $\left\{u,v\right\}\in E(G)$, decimos que $u$ y $v$ son adyacentes. Se denota tal adyacencia por $u\sim v$.
\end{definition}

\begin{definition}
Sea $G$ una gráfica. Una subgráfica de $G$ es una gráfica $H$ tal que $V(H)\subset V(G)$ y $E(H)\subset E(G)$.
\end{definition}

\begin{definition}
Sea $G$ una gráfica y $H$ una subgráfica de $G$. $H$ es una subgráfica inducida si para todo $u,v\in V(H)$ tales que $u\sim v$ en $G$ entonces $u\sim v$ en $H$.
\end{definition}

Se entiende que una subgráfica completa $H_n$ tiene cada par de sus $n$ vértices adyacentes. Dicha gráfica completa es maximal si no existe otro vértice en la gráfica tal que forme una completa más grande. Lo anterior da paso a la siguiente definición.

\begin{definition}[\citealt{Harary:1969}]
Un clan de $G$ es una subgráfica completa maximal. 
\end{definition}

\begin{definition}[\citealt{Roberts:1971}]
Dado $G$ una gráfica, sean $C_1, C_2, \dots, C_n $ sus clanes. Definimos $H'$ mediante $ V(H') = \{C_1, C_2, \dots, C_n\}$ y $\left\{C_i, C_j\right\}\in E(H')$ si y solo si $i \neq j$ y $C_i \cap C_j \neq \emptyset$.  
Entonces, llamamos a $H'$ como la gráfica de clanes de $G$ y escribimos $H'=K(G)$.
\end{definition}

\begin{definition}[\citealt{Alcon:2006}]
Sea $G$ una gráfica y $v \in V(G)$. Como es habitual, $G-v$ denota la gráfica inducida por $V(G)\setminus \{v\}$.  

El vértice $v$ es clan crítico si $K(G)\neq K(G-v)$. Una gráfica $G$ es clan crítica si cada uno de sus vértices es crítico.
\end{definition}



\section*{Resultados}

\begin{theorem}
	Sea $G$ una gráfica clan crítica tal que su gráfica de clanes es $k_3$, entonces $G$ es la gráfica 4,3,3 o 5,7,1 (ver en \cite{Harary:1969} el apéndice de gráficas).
\end{theorem}
\begin{proof}
Sea $G$ una gráfica clan crítica tal que $K(G)=k_3$.
\end{proof}











\bibliography{Ref}
\bibliographystyle{apalike}

\end{document}